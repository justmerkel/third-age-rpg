\begin{multicols}{2}
\section*{General} %%%%%%%%%%%%%%%%GENERAL
\addcontentsline{toc}{section}{General}
\begin{mercHeading}
Multiclassing
\end{mercHeading}
Multiclassing has 2 flavors.\\
\textbf{Starting with multiple classes:} A player wanting multiple classes should split xp betwixt each class and divide hp by the number of classes. Take the max of \textbf{Magic}, max of \textbf{Divinity}, and max of \textbf{Attack Bonus} \\
\textbf{Adding a class:} A player that adds a class later on, such as a fighter dabbling in magic-user, must spend all new xp on the new class until it reaches the level of the other class.
\begin{mercHeading}
When to make an attribute roll
\end{mercHeading}
Any time a character attempts to do something where the outcome is uncertain. For example a character may try to hide from a patrol of goblins, the referee would call for a \textbf{Dexterity} attribute roll.
\begin{mercHeading}
Converting Saving Throws
\end{mercHeading}
The saving throws from classic material are Death/Poision, Wands, Paralysis/Petrify, Breath attack, Spells/Rods/Staves. The Third Age rpg uses attribute rolls instead of saving throws and each are converted to the following.
\begin{itemize}
\setlength\itemsep{0em}
	\item \textbf{Death/Poison} is converted to \textbf{Body}
	\item \textbf{Wands} is converted to the spell's target attribute
	\item \textbf{Paralysis/Petrify} is converted to \textbf{Spirit}
	\item \textbf{Breath Attack} is converted to \textbf{Dexterity}
	\item \textbf{Spells/Rods/Staves} is converted to the spell's target attribute
\end{itemize}


\begin{mercHeading}
Languages
\end{mercHeading}

Every character in the Third Age knows the common tongue. For each point of \textbf{Mind} above 0, the character knows an additional language from this list. If a character has -3, they are illiterate. 
\begin{itemize}
\setlength\itemsep{0em}
	\item Elvish
	\item Dwarvish
	\item Orcish
	\item Gnomish
	\item Draconic
	\item Abyssal
	\item Goblin
	\item Infernal
	\item Celestial
\end{itemize}

\begin{mercHeading}
Using Non-Proficient Armor/Weapons
\end{mercHeading}
A character may use weapons and armor that are not listed in one of their classes. In this case, every action using it will require an attribute roll.
\section*{Encounters}%%%%%%%%%%%%%%%%%%%%%% ENCOUNTERS
\addcontentsline{toc}{subsection}{Encounters}
\begin{mercHeading}
Surprise
\end{mercHeading}

Before an encounter begins or as a method to adjudicate group stealth, the referee should perform a surprise roll 1d6 for players and monsters. On 1-2 the opposing side is surprised and cannot act for the first round of combat.

\begin{mercHeading}
Reaction Rolls
\end{mercHeading}
Anytime a random encounter occurs, the referee should roll 2d10 modified by the party leader's \textbf{Charisma} attribute.%Reac
\begin{table}[H]

\begin{center}
\Large
\rowcolors{2}{gray!30}{greg}
\begin{tabular}{ c  c  }


\textbf{
Roll Result} &\textbf{ Reaction}\\
\bottomrule
\bottomrule

6- & Immediate Hostility \\

7-10 &  Unfriendly\\

11-14 & Neutral, Suspicious \\

15-18 & Uninterested \\

19+ & Friendly \\


\end{tabular}
\end{center}
\label{table:RollingAttributes}
\vspace{-1cm}
\end{table}


\section*{Dungeon}
\addcontentsline{toc}{subsection}{Dungeon}
\begin{mercHeading}
Dungeon Turn
\end{mercHeading}
A dungeon turn is a 10 min segment of time that the players can use to perform actions. The flow of each turn is a 3-step process.
\begin{itemize}
\setlength\itemsep{0em}
	\item \textbf{Declaring Actions:} Players decide what they do (Hide, Search, Listen, Move, Interact,...)
	\item \textbf{Describe what happened:} Determine if an attribute roll is required or the result of the action
	\item \textbf{Wandering Monster:} Referee rolls 1d6 and on a 1, a random encounter is rolled
\end{itemize}
\begin{mercHeading}
Random Encounters
\end{mercHeading}
\textbf{Random Encounters} or \textbf{Wandering Monster} is typically a table of monsters that you would find in the dungeon. The monster rolled is not necessarily hostile and the referee should roll a \textbf{Reaction Roll} to determine their disposition. Depending on the dungeon this check should occur every turn to every three turns.

\begin{mercHeading}
Movement in Dungeons
\end{mercHeading}
The Movement in a dungeon is 60' (12 squares), this may seem like a small number but it is assumed that the players are actively checking every section (preferrably with a 10' pole). This also means that any trap or hazard should be telegraphed. If the players wish to move faster, they will not get the forewarning. This also means backtracking should involve fewer turns. 


\section*{Combat} %%%%%%%%%%%%%%%%%%%%%%%%%%%%%%%%COMBAT
\addcontentsline{toc}{section}{Combat}
\begin{mercHeading}
Combat Steps
\end{mercHeading}
\begin{enumerate}
\setlength\itemsep{0em}
	\item Declare Movement in Melee
	\item Roll Initative
	\item Take your turn
	\begin{enumerate}
	\item Morale
	\item Movement
	\item Attack/Spell/Improvised Action
	\end{enumerate}
\end{enumerate}
\begin{mercHeading}
Initiative 
\end{mercHeading}
\textbf{Group Initiative:} The default method is to first declare if a character will move out of melee. Then, roll 1d6 for each group in combat, higher number better. On ties, both groups go simultaneously.\\
\textbf{Individual Initiative:} An alternative method is to roll 1d6 for each combatant (optionally modified by Dexterity)

\begin{mercHeading}
Improvised Action 
\end{mercHeading}
Sometimes there is something a player wants to do that is not an attack or spell. In this case the referee should decide on an attribute and apply difficulty based on what the player is trying to do. For example, a player may attempt a grapple with a body attribute roll.

\begin{mercHeading}
Reactions
\end{mercHeading}
A player can in some cases perform an action when it's not their turn. When a player takes a reaction, (except Counterspell) they use a stamina point.\\
\textbf{Counter:} When a creature fails an attack against you, spend 1 stamina point to roll weapon damage.
\textbf{Parry:} A character can interpose their weapon or shield to impose a -2 to an attack roll\\
\textbf{Counterspell:} A character that has the same spell memorized that is being cast can make their own \textbf{Mind} attribute roll that lowers the opposing caster's level of success.\\
\textbf{Mr. President Get Down:} A character can take the damage of an attack near them but must fall prone.


\begin{mercHeading}
Movement
\end{mercHeading}
In combat, characters can move 40' (8 squares) per round in combat. \\
\textbf{Engaged in Melee:} If a character begins their turn adjacent to a enemy and uses movement, they receive a -2 penalty to AC for the round.

\begin{mercHeading}
Attack Rolls
\end{mercHeading}
\label{label:atkroll}
Attack rolls are 2d10 + attribute + attack bonus that are successful if the result is greater than or equal to the target's success armor class. The target takes an amount of damage equal to the weapon die. On a partial success (< success AC and >= partial AC), the target takes half damage rounded down. 
The target can spend a point of stamina to counter at half damage. On a failure, the target can spend a point of stamina to counter for full damage. \\
\textbf{Melee Attacks:} Add Body to attack and damage rolls \\
\textbf{Ranged Attacks:} Add Dexterity to attack rolls

\begin{mercHeading}
Two-Weapon Fighting and One Hand Free
\end{mercHeading}
\textbf{Two-Weapons:} A character that is wielding 2 weapons of d6 damage or less may make 2 attacks, one at -2 and the second at -4. \\
\textbf{One Hand Free:} A character with one hand free adds +1 to attack rolls.


\begin{mercHeading}
Morale and Loyalty
\end{mercHeading}
\textbf{Morale:} Most monsters have a morale rating this serves 2 purposes. The first, when the monsters have sustained multiple casualites the referee should check morale. Roll 2d6 and if the result is above the monster's morale rating, they flee. The second is when a character attempts to intimidate or persuade monsters to surrender. The player needs to roll above the monster's morale rating on a \textbf{Charisma} attribute roll.\\
\textbf{Loyalty:} Loyalty is similar to morale but for \textbf{Retainers}. This is a special stat that is tracked for each of the player's retainer that starts at 7 modified by the player's \textbf{Charisma}. Actions that cause harm to the retainer may lower this stat and actions that benefit the retainer (such as giving a magic item) may increase this stat. 