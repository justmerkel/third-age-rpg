\documentclass[18pt]{article}

\usepackage{multicol}
\usepackage{color}
\usepackage[many]{tcolorbox}
\usepackage{lipsum}
\usepackage{setspace}
%\usepackage{float}
\usepackage{booktabs}
\usepackage{listings}
\usepackage{tabularx}
\usepackage{array}

\newcolumntype{P}[1]{>{\centering\arraybackslash}p{#1}}
\newcolumntype{Y}{>{\centering\arraybackslash}X}

\newcolumntype{b}{>{\centering\arraybackslash}X}
\newcolumntype{L}{X}
\newcolumntype{Q}{>{\centering\arraybackslash\hsize=.2\hsize}X}
\newcolumntype{s}{>{\centering\arraybackslash\hsize=.5\hsize}X}
\newcolumntype{S}{>{\centering\arraybackslash\hsize=.4\hsize}X}
\newcolumntype{h}{>{\centering\arraybackslash\hsize=.6\hsize}X}
\newcolumntype{t}{>{\centering\arraybackslash\hsize=1\hsize}X}

\usepackage{xcolor,colortbl}
\usepackage[a4paper, margin=0.5in,footskip=0.5in, bottom=1in]{geometry}
\tcbuselibrary{listings,breakable}
\usetikzlibrary{calc}

\definecolor{myblue}{RGB}{0,163,243}
\definecolor{greg}{RGB}{164,216,216}
\definecolor{table2}{RGB}{216, 164, 164}
%\definecolor{table2}{RGB}{216, 190, 164}
\newcommand{\mhead}[1]{\paragraph{#1}\mbox{}\\\\}
\usepackage{titlesec}
\usepackage{lmodern}
\usepackage{booktabs}
\titleformat*{\section}{\Huge\bfseries}
\titleformat*{\subsection}{\LARGE\bfseries}
\titleformat*{\subsubsection}{\large\bfseries}
\titleformat*{\paragraph}{\large\bfseries}
\titleformat*{\subparagraph}{\large\bfseries}
\tcbset{mystyle/.style={
  breakable,
  enhanced,
  outer arc=0pt,
  arc=0pt,
  colframe=greg,
  colback=greg,
  attach boxed title to top left,
  boxed title style={
    colback=greg,
    outer arc=0pt,
    arc=0pt,
    },
  title=Example~\thetcbcounter,
  fonttitle=\sffamily
  }
}

\usepackage{floatrow}
\DeclareFloatFont{sans}{\sffamily \centering}% "scriptsize" is defined by floatrow, "tiny" not
\floatsetup[table]{font=sans}

\newtcolorbox{mercHeading}{
  width=\columnwidth,
  fontupper = \huge \sffamily,
  colback={greg},
boxsep=5pt,left=0pt,right=0pt,top=0pt,bottom=0pt,
  frame hidden,
  boxrule =0pt,
  arc = 0mm,
  outer arc =0mm,
  colupper = black
}

\newtcolorbox{mercClassInfo}{
  width=\columnwidth,
  fontupper = \large \sffamily,
  colback={greg},
boxsep=5pt,left=0pt,right=0pt,top=0pt,bottom=0pt,
  frame hidden,
  boxrule =0pt,
  arc = 0mm,
  outer arc =0mm,
  colupper = black
}



\title{\fontsize{70}{80} \selectfont The Third Age Rules Handbook}
\author{J. R. Merkel}
\date{}
\begin{document}
\sffamily
\large
\frenchspacing
\singlespacing
\parskip=0pt
\parindent=0pt
\lstset{aboveskip=0pt, belowskip=0pt}
%Paragraph header on newline
%\newcommand{\h2}[1]{\paragraph{#1}\mbox{}\\}


\maketitle
\thispagestyle{empty}
\pagebreak
\thispagestyle{empty}
\tableofcontents
\pagebreak
 \pagenumbering{arabic}
\section*{Introduction}
\addcontentsline{toc}{section}{Introduction}

\subsection*{Design Goals}
\begin{itemize}
	\item Maintain compatibility and tone from Old-school material.
        \item Revise pain points of B/X
        \item Incorporate 3 tier of outcomes from pbta
\end{itemize}
%%%%%%%%%%%Begin Actual Content%%%%%%%%%%%%%%%
\begin{multicols}{2}
\section*{Character}
\addcontentsline{toc}{section}{Character}
\subsection*{Creating a Character}
A Third Age Character has 5 
steps.
\begin{enumerate}
\setlength\itemsep{0em}
	\item Generate attributes \emph{pg. \pageref{label:Attributes}}
	\item Decide on a class \emph{pg. \pageref{label:Class}}
	\item Roll hit points and starting gold, then fill out attack bonus, magic/divine, languages, and spells
	\item Purchase equipment and fill out Armor Class \emph{pg. \pageref{section:Equipment}}
	\item Create a name and backstory/personality %TODO Personality/Qualities from SR???
\end{enumerate}


\subsection*{Attributes}
\addcontentsline{toc}{subsection}{Attributes}
\label{label:Attributes}
The Third Age RPG uses 5 main stats that define a character's strengths and weaknesses ranging from -3 to +3. 

\begin{mercHeading}
Body
\end{mercHeading}
The measure of physical ability. The primary attribute of Fighting-men.

\begin{itemize}
	\setlength\itemsep{0em}
	\item \textbf{Hit Points:} Increases hit points by Body each level (Min 1)
	\item \textbf{Melee Attack Bonus:} Increases to-hit and damage rolls of melee attacks by Body
\end{itemize}

\begin{mercHeading}
Mind
\end{mercHeading}
The measure of mental acuity. The primary attribute of Magic-Users.

\begin{itemize}
\setlength\itemsep{0em}
	\item \textbf{Languages:} Increase the number of languages known by Mind
	\item \textbf{Spellcasting:} Increases spellcasting checks by Mind
\end{itemize}

\begin{mercHeading}
Spirit
\end{mercHeading}
The measure of will and belief. The primary attribute of Clerics

\begin{itemize}
\setlength\itemsep{0em}
	\item \textbf{Miracles:} Increases miracle checks by Spirit
\end{itemize}

\begin{mercHeading}
Charisma
\end{mercHeading}
The measure of personality and charm. The primary attribute of TBD? nothing

\begin{itemize}
\setlength\itemsep{0em}
	\item \textbf{Reaction:} Increase reaction roll result by Charisma
	\item \textbf{Retainers:} Increase max number of retainers and their morale check by Charisma
\end{itemize}

\begin{mercHeading}
Dexterity
\end{mercHeading}

\begin{itemize}
\setlength\itemsep{0em}
	\item \textbf{Armor Class:} Increase Armor Class (AC) by Dexterity
	\item \textbf{Ranged Attack Bonus:} Increases to-hit of ranged attacks by Dexterity
\end{itemize}
The measure of deftness and agility. The primary attribute of Thieves

\begin{mercHeading}
Assigning Attributes
\end{mercHeading}

Roll 3d6 for each of the 5 Attributes in order accoring to the following table. Note, this does produce more powerful characters than traditional OSR stat generation.


\begin{table}[H]

\begin{center}
\Large
\rowcolors{2}{gray!30}{greg}
\begin{tabular}{ c  c  }


\textbf{
Roll Result} &\textbf{ Bonus}\\
\bottomrule
\bottomrule

3 & -3 \\

4-5 & -2 \\

6-7 & -1 \\

8-9 & 0 \\

10-13 & +1 \\

14-16 & +2 \\

17-18 & +3 \\

\end{tabular}
\end{center}
\label{table:RollingAttributes}
\vspace{-1cm}
\end{table}
\vspace{1cm}
\begin{mercHeading}
Attribute Roll
\end{mercHeading}

An \textbf{Attribute Roll} is 2d6 + attribute. On a result of 6 or less, the roll is a failure. On a result of 7-9, the roll is a partial success. On a result of 10+ the roll is considered a success.\\
\textbf{Modifying Attribute Rolls} A referee may decide any attribute roll to be easier (add up to +3 to a result) or more difficult (subtract up to 3). \\
\textbf{Advantage/Disadvantage} Another way to modify rolls is to use advantage best 2 out of 3 d6 or disadvantage worst 2 out of 3 d6.
\subsection*{Other Stats}
There are other stats on the character sheet that are dependent on the class.

\begin{mercHeading}
Health and Hit Die
\end{mercHeading}
Health (\textbf{HP}) is the number of points of damage a character can take before death. This stat is generated by the \textbf{Hit Dice} of the class by rolling (At first level take the Maximum). For monsters, this is always a d8 and the number of hit dice affects how some stats and saves work.

\begin{mercHeading}
Attack Bonus
\end{mercHeading}
The Attack Bonus is added to the attack roll as described on pg \pageref{label:atkroll}.

\begin{mercHeading}
Magic/Divine
\end{mercHeading}
This is the stat that governs spells or miracles respectively. This is explained in more detail on the class pages and general magic rules on pg \pageref{section:GenMagicRules}.

\begin{mercHeading}
Armor Class
\end{mercHeading}
Armor Class (\textbf{AC}) is the measure of a character's ability to take blows. This is determined by armor + dexterity. To convert old school material use 19-AC = Third Age AC.

\begin{mercHeading}
Experience
\end{mercHeading}
Experience (\textbf{XP}) is the measure of the amount of cumulative \emph{experience} a character has had adventuring. This number increases by 1 for each gold piece recovered and a certain amount per monster determined by the referee.
\section*{Class}
\addcontentsline{toc}{section}{Class}
\subsection*{Class List}
\label{label:Class}
\begin{itemize}
	\item Magic-User
	\item Fighting-Man
	\item Thief
	\item Cleric
\end{itemize}

\end{multicols}
\section*{Cleric}%%%%%%%%%%%%%%%CLERIC
\addcontentsline{toc}{subsection}{Cleric}
\begin{multicols}{3}
\begin{mercClassInfo}
\textbf{Casting Attribute:} Spirit\\
\textbf{Armor:} All\\
\textbf{Weapons:} Weapons with Blunt Property
\end{mercClassInfo}
\end{multicols}
\subsubsection*{Level Progression}
\begin{table}[H]

\begin{center}

\Large
\rowcolors{2}{gray!30}{greg}
\centering
\begin{tabularx}{\textwidth}{>{\centering\arraybackslash\hsize=.2\hsize}X 
>{\centering\arraybackslash\hsize=.2\hsize}X
>{\centering\arraybackslash\hsize=.3\hsize}X
>{\centering\arraybackslash\hsize=.2\hsize}X
>{\centering\arraybackslash\hsize=.55\hsize}X
>{\centering\arraybackslash\hsize=.55\hsize}X}



\hiderowcolors
 \textbf{
Level}& \textbf{XP} &\textbf{Hit Dice} &\textbf{Divine} &  \textbf{Max Miracle Power} & \textbf {Attack Bonus}\\
\hline
\hline
%\bottomrule
\showrowcolors
\centering
1 &0&  1d6  & 1 & 1 & 0\\

2 &2000&  2d6 & 2 & 1 & 1 \\

3 &4000& 3d6 & 3 & 2 &1\\

4 &8000& 4d6 & 4 & 2 & 2\\

5 &16000& 5d6 & 5 & 3 & 2\\

6 &32000& 6d6 &6 & 3 & 3\\

7 &64000& 7d6 & 7 & 4 & 3\\

\end{tabularx}
\end{center}
\label{table:Cleric}
\end{table}

\begin{multicols}{2}%%%Cleric data

\begin{mercHeading}
Divine Spellcasting
\end{mercHeading}

See  \textbf{pg \pageref{section:GenMagicRules} Magic} for general spellcasting rules. \\
\textbf{Miracles Known:} A first-level Cleric knows a number of miracles equal to \textbf{Spirit} (Min 1.) This number is split between 0th level miracles (cantrips) and 1st level miracles. When a magic user gains a level, they learn an additional miracle of a level less than or equal to the max miracle power of the table. \vspace{2pt}\\ 
\textbf{Miracle Scrolls:}
A Cleric is able to cast any Miracle scroll (For every level above max miracle power adds -1 to the attribute roll). Additionally, a Cleric can learn a miracle from a scroll after spending 1 day per miracle power in prayer.

\begin{mercHeading}
Turn Undead
\end{mercHeading}
A Cleric has a special ability with their holy symbol that can repel or destroy Undead creatures. The cleric makes a \textbf{Spirit} attribute roll that turns Undead up to Cleric Level HD and destroys half Cleric Level HD on a success. On a partial, turns half Cleric Level HD, and on a failure does nothing. \\


\end{multicols}

\section*{Thief}%%%%%%%%%%%%%%%%%%%%%THIEF
\addcontentsline{toc}{subsection}{Thief}
\begin{multicols}{3}
\begin{mercClassInfo}
\textbf{Casting Attribute:} None\\
\textbf{Armor:} Light\\
\textbf{Weapons:} All
\end{mercClassInfo}
\end{multicols}
\subsubsection*{Level Progression}
\begin{table}[H]

\begin{center}

\Large
\rowcolors{2}{greg}{gray!30}
\centering
\begin{tabularx}{\textwidth}{>{\centering\arraybackslash\hsize=.33\hsize}X 
>{\centering\arraybackslash\hsize=.33\hsize}X
>{\centering\arraybackslash\hsize=.34\hsize}X
>{\centering\arraybackslash\hsize=.5\hsize}X
>{\centering\arraybackslash\hsize=.5\hsize}X}

\hiderowcolors


 \textbf{
Level}& \textbf{XP} &\textbf{Hit Dice} & \textbf {Attack Bonus} & \textbf{Thief Expertise}\\
\bottomrule
\bottomrule
\showrowcolors
\centering
1 &0&  1d6  & 1 & 1\\

2 &1500&  2d6 & 1 & 1 \\

3 &3000& 3d6 & 2 & 1\\

4 &6000& 4d6 & 2 & 2\\

5 &12000& 5d6 & 3 & 2\\

6 &24000& 6d6 & 3 & 2\\

7 &48000& 7d6 & 4 & 3\\

\end{tabularx}
\end{center}
\label{table:Thief}
\end{table}

\begin{multicols}{2} %%% Thief Data

\begin{mercHeading}
Backstab 
\end{mercHeading}

When the thief successfully sneaks behind an enemy unseen, they attack with +4 bonus to hit and deal double damage.

\begin{mercHeading}
Thief Expertise
\end{mercHeading}
A thief character adds \textbf{Thief Expertise} to any attribute roll that involves the following
\begin{itemize}
	\setlength\itemsep{0em}
	\item Lockpick
	\item Hiding
	\item Moving Silently
	\item Climbing
	\item Pickpocketing
	
\end{itemize}

\end{multicols}

\section*{Fighting-Man}%%%%%%%%%%%%%%%%%%%%Fighting-Man
\addcontentsline{toc}{subsection}{Fighting-Man}
\begin{multicols}{3}
\begin{mercClassInfo}
\textbf{Casting Attribute:} None\\
\textbf{Armor:} All\\
\textbf{Weapons:} All
\end{mercClassInfo}
\end{multicols}

\subsubsection*{Level Progression}
\vspace{-10pt}

\begin{table}[H]

\begin{center}

\Large
\rowcolors{2}{gray!30}{greg}
\centering
\begin{tabularx}{\textwidth}{s s s s}

\hiderowcolors


 \textbf{
Level}& \textbf{XP} &\textbf{Hit Dice} & \textbf {Attack Bonus}\\
\bottomrule
\bottomrule
\showrowcolors
\centering
1 &0&  1d8  & 1 \\

2 &2000&  2d8 & 2  \\

3 &4000& 3d8 & 3 \\

4 &8000& 4d8 & 4 \\

5 &16000& 5d8 & 5 \\

6 &32000& 6d8 &6 \\

7 &64000& 7d8 & 7 \\

\end{tabularx}
\end{center}
\label{table:Fighter}
\end{table}

\begin{multicols}{2}
You're a chad who doesn't need abilities %%%TODO WHAT IS A FIGHTER
\end{multicols}

\section*{Magic-User}%%%%%%%%%%%%%%%%%MAGIC USER
\addcontentsline{toc}{subsection}{Magic-User}
\begin{multicols}{3}
\begin{mercClassInfo}
\textbf{Casting Attribute:} Mind\\
\textbf{Armor:} None\\
\textbf{Weapons:} Dagger, Crossbow
\end{mercClassInfo}
\end{multicols}
\subsubsection*{Level Progression}
\vspace{-10pt}

\begin{table}[H]

\begin{center}

\Large
\rowcolors{2}{gray!30}{greg}
\centering
\begin{tabularx}{\textwidth}{>{\centering\arraybackslash\hsize=.2\hsize}X 
>{\centering\arraybackslash\hsize=.2\hsize}X
>{\centering\arraybackslash\hsize=.3\hsize}X
>{\centering\arraybackslash\hsize=.2\hsize}X
>{\centering\arraybackslash\hsize=.55\hsize}X
>{\centering\arraybackslash\hsize=.55\hsize}X}

\hiderowcolors


 \textbf{
Level}& \textbf{XP} &\textbf{Hit Dice} &\textbf{Magic} &  \textbf{Max Spell Power} & \textbf {Attack Bonus}\\
\bottomrule
\bottomrule
\showrowcolors
\centering
1 &0&  1d4  & 1 & 1 & 0\\

2 &2500&  2d4 & 2 & 1 & 0 \\

3 &5000& 3d4 & 3 & 2 &1\\

4 &10000& 4d4 & 4 & 2 & 1\\

5 &20000& 5d4 & 5 & 3 & 1\\

6 &40000& 6d4 &6 & 3 & 2\\

7 &80000& 7d4 & 7 & 4 & 2\\

\end{tabularx}
\end{center}
\label{table:MagicUser}
\end{table}
\begin{multicols}{2}
\begin{mercHeading}
Arcane Spellcasting
\end{mercHeading}
See  \textbf{pg \pageref{section:GenMagicRules} Magic} for general spellcasting rules. \\
\textbf{Spells Known:} A first-level magic user knows a number of spells equal to \textbf{Mind} (Min 1.) This number is split between 0th level spells (cantrips) and 1st level spells. When a magic user gains a level, they learn an additional spell of a level less than or equal to the max spell power of the table. \vspace{2pt}\\ 
\textbf{Spell Scrolls:}
A magic user is able to cast any spell scroll. Additionally, a magic user can learn a spell from a scroll after spending 1 day per spell power in study.\\

\section*{General} %%%%%%%%%%%%%%%%GENERAL
\addcontentsline{toc}{section}{General}
\begin{mercHeading}
Multiclassing
\end{mercHeading}
Multiclassing has 2 flavors.\\
\textbf{Starting with multiple classes:} A player wanting multiple classes should split xp betwixt each class and divide hp by the number of classes. Take the max of \textbf{Magic}, max of \textbf{Divine}, and max of \textbf{Attack Bonus} \\
\textbf{Adding a class:} A player that adds a class later on, such as a fighter dabbling in magic-user, must spend all new xp on the new class until it reaches the level of the other class.
\begin{mercHeading}
When to make an attribute roll
\end{mercHeading}
Any time a character attempts to do something where the outcome is uncertain. For example a character may try to hide from a patrol of goblins, the referee would call for a \textbf{Dexterity} attribute roll.
\begin{mercHeading}
Converting Saving Throws
\end{mercHeading}
The saving throws from classic material are Death/Poision, Wands, Paralysis/Petrify, Breath attack, Spells/Rods/Staves. The Third Age rpg uses attribute rolls instead of saving throws and each are converted to the following.
\begin{itemize}
\setlength\itemsep{0em}
	\item \textbf{Death/Poison} is converted to \textbf{Body}
	\item \textbf{Wands} is converted to the spell's target attribute
	\item \textbf{Paralysis/Petrify} is converted to \textbf{Spirit}
	\item \textbf{Breath Attack} is converted to \textbf{Dexterity}
	\item \textbf{Spells/Rods/Staves} is converted to the spell's target attribute
\end{itemize}


\begin{mercHeading}
Languages
\end{mercHeading}

Every character in the Third Age knows the common tongue. For each point of \textbf{Mind} above 0, the character knows an additional language from this list. If a character has -3, they are illiterate. 
\begin{itemize}
\setlength\itemsep{0em}
	\item Elvish
	\item Dwarvish
	\item Orcish
	\item Gnomish
	\item Draconic
	\item Abyssal
	\item Goblin
	\item Infernal
	\item Celestial
\end{itemize}

\begin{mercHeading}
Using Non-Proficient Armor/Weapons
\end{mercHeading}
A character may use weapons and armor that are not listed in one of their classes. In this case, every action using it will require an attribute roll.
\section*{Encounters}%%%%%%%%%%%%%%%%%%%%%% ENCOUNTERS
\addcontentsline{toc}{subsection}{Encounters}
\begin{mercHeading}
Surprise
\end{mercHeading}

Before an encounter begins or as a method to adjudicate group stealth, the referee should perform a surprise roll 1d6 for players and monsters. On 1-2 the opposing side is surprised and cannot act for the first round of combat.

\begin{mercHeading}
Reaction Rolls
\end{mercHeading}
Anytime a random encounter occurs, the referee should roll 2d10 modified by the party leader's \textbf{Charisma} attribute.%Reac
\begin{table}[H]

\begin{center}
\Large
\rowcolors{2}{gray!30}{greg}
\begin{tabular}{ c  c  }


\textbf{
Roll Result} &\textbf{ Reaction}\\
\bottomrule
\bottomrule

6- & Immediate Hostility \\

7-10 &  Unfriendly\\

11-14 & Neutral, Suspicious \\

15-18 & Uninterested \\

19+ & Friendly \\


\end{tabular}
\end{center}
\label{table:RollingAttributes}
\vspace{-1cm}
\end{table}


\section*{Dungeon}
\addcontentsline{toc}{subsection}{Dungeon}
\begin{mercHeading}
Dungeon Turn
\end{mercHeading}
A dungeon turn is a 10 min segment of time that the players can use to perform actions. The flow of each turn is a 3-step process.
\begin{itemize}
\setlength\itemsep{0em}
	\item \textbf{Declaring Actions:} Players decide what they do (Hide, Search, Listen, Move, Interact,...)
	\item \textbf{Describe what happened:} Determine if an attribute roll is required or the result of the action
	\item \textbf{Wandering Monster:} Referee rolls 1d6 and on a 1, a random encounter is rolled
\end{itemize}
\begin{mercHeading}
Random Encounters
\end{mercHeading}
\textbf{Random Encounters} or \textbf{Wandering Monster} is typically a table of monsters that you would find in the dungeon. The monster rolled is not necessarily hostile and the referee should roll a \textbf{Reaction Roll} to determine their disposition. Depending on the dungeon this check should occur every turn to every three turns.

\begin{mercHeading}
Movement in Dungeons
\end{mercHeading}
The Movement in a dungeon is 60' (12 squares), this may seem like a small number but it is assumed that the players are actively checking every section (preferrably with a 10' pole). This also means that any trap or hazard should be telegraphed. If the players wish to move faster, they will not get the forewarning. This also means backtracking should involve fewer turns. 


\section*{Combat} %%%%%%%%%%%%%%%%%%%%%%%%%%%%%%%%COMBAT
\addcontentsline{toc}{section}{Combat}
\begin{mercHeading}
Combat Steps
\end{mercHeading}
\begin{enumerate}
\setlength\itemsep{0em}
	\item Declare Movement in Melee
	\item Roll Initative
	\item Take your turn
	\begin{enumerate}
	\item Morale
	\item Movement
	\item Attack/Spell/Improvised Action
	\end{enumerate}
\end{enumerate}
\begin{mercHeading}
Initiative 
\end{mercHeading}
\textbf{Group Initiative:} The default method is to first declare if a character will move out of melee. Then, roll 1d6 for each group in combat, higher number better. On ties, both groups go simultaneously.\\
\textbf{Individual Initiative:} An alternative method is to roll 1d6 for each combatant (optionally modified by Dexterity)

\begin{mercHeading}
Improvised Action 
\end{mercHeading}
Sometimes there is something a player wants to do that is not an attack or spell. In this case the referee should decide on an attribute and apply difficulty based on what the player is trying to do. For example, a player may attempt a grapple with a body attribute roll.

\begin{mercHeading}
Reactions
\end{mercHeading}
A player can in some cases perform an action when it's not their turn. When a player takes a reaction, they forfeit their next turn.\\
\textbf{Parry:} A character can interpose their weapon or shield to impose a -2 to an attack roll\\
\textbf{Counterspell:} A character that has the same spell memorized that is being cast can make their own \textbf{Mind} attribute roll that lowers the opposing caster's level of success.\\
\textbf{Mr. President Get Down:} A character can take the damage of an attack near them but must fall prone.

\begin{mercHeading}
Movement
\end{mercHeading}
In combat, characters can move 40' (8 squares) per round in combat. \\
\textbf{Engaged in Melee:} If a character begins their turn adjacent to a enemy and uses movement, they receive a -2 penalty to AC for the round.

\begin{mercHeading}
Attack Rolls
\end{mercHeading}
\label{label:atkroll}
Attack rolls are 1d20 + attribute + attack bonus that are successful if the result is greater than or equal to the target's armor class. The target takes an amount of damage equal to the weapon die.\\
\textbf{Melee Attacks:} Add Body to attack and damage rolls \\
\textbf{Ranged Attacks:} Add Dexterity to attack rolls

\begin{mercHeading}
Two-Weapon Fighting and One Hand Free
\end{mercHeading}
\textbf{Two-Weapons:} A character that is wielding 2 weapons of d6 damage or less may make 2 attacks, one at -2 and the second at -4. \\
\textbf{One Hand Free:} A character with one hand free adds +1 to attack rolls.


\begin{mercHeading}
Morale and Loyalty
\end{mercHeading}
\textbf{Morale:} Most monsters have a morale rating this serves 2 purposes. The first, when the monsters have sustained multiple casualites the referee should check morale. Roll 2d6 and if the result is above the monster's morale rating, they flee. The second is when a character attempts to intimidate or persuade monsters to surrender. The player needs to roll above the monster's morale rating on a \textbf{Charisma} attribute roll.\\
\textbf{Loyalty:} Loyalty is similar to morale but for \textbf{Retainers}. This is a special stat that is tracked for each of the player's retainer that starts at 7 modified by the player's \textbf{Charisma}. Actions that cause harm to the retainer may lower this stat and actions that benefit the retainer (such as giving a magic item) may increase this stat. 

\section*{Magic}%%%%%%%%%%%%%%%%%%%%%%%%%%%%%%%%MAGIC
\addcontentsline{toc}{section}{Magic}
\label{section:GenMagicRules}
Every character capable of casting spells or miracles has a \textbf{Magic} or \textbf{Divine} stat. This stat acts as maximum amount of wild magic energy a character can withstand before being unable to cast. That energy is expressed as a temperature gauge (known as \textbf{Wild Magic Gauge}) on a character's sheet with a number of segments equal to the stat. If a spell would exceed this gauge on a failure, the spell cannot be cast.
\begin{mercHeading}
{Adding to Wild Magic Gauge}
\end{mercHeading}
When a character casts a spell from a given class, they must make an attribute roll of the class's casting attribute. On a failure, the spell fizzles and has either no or reduced effect, and the spell's level is added to the wild magic gauge. On a partial success, the spell executes as normal, but the spell's level is added to the wild magic gauge. On a success, the spell executes as normal. Cantrips are a special type of spell that do not add to the gauge on a partial success.\vspace{2pt}\\
  \textbf{Full Wild Magic Gauge:}
Once the wild magic gauge is filled, the spell has an additional wild magic effect. This is determined by rolling on the Wild Magic table on pg \pageref{table:WildMagic} or an effect the referee thinks is thematic.
\begin{mercHeading}
Concentration
\end{mercHeading}

Spells that specify concentration or require multiple rounds to cast, uses the caster's action every round until the spell ends or is cast. When a caster is hit while concentrating, they must make a \textbf{Body} attribute roll to maintain the spell. On a failure, the spell fizzles and ends. On a partial success, the caster can choose to roll on the wild magic table or end the spell. On a success, nothing happens.

\end{multicols}

\section*{Wild Magic}%%%%%%%%%%%%%%%%%%%Wild Magic
\addcontentsline{toc}{subsection}{Wild Magic}
\begin{table}[H]

\begin{center}

\large
\rowcolors{2}{gray!30}{greg}

\begin{tabularx}{\textwidth}{s 
>{\arraybackslash\hsize=1.5\hsize}X}

\hiderowcolors


 \textbf{
1d100 Result}& \centering\arraybackslash\textbf{Wild Magic Effect}\\
\bottomrule
\bottomrule
\showrowcolors

1-10 & Wild Magic Effect as described in the spell (spell specific) \\
11-12 & Caster turns into a potted plant for 1d3 dungeon turns \\
13-14 & Caster and target swap places, if no target, random \\
15-16 & Caster turns invisible \\
17-18 & Caster Ages 1d4 years \\
19-20 & Caster De-ages 1d4 years \\
21-22 & Caster speaks a deep secret \\
23-24 & Vegetation within 30ft of the caster withers \\
25-26 & Caster gains telepathy for the next hour \\
27-28 & Caster's skin turns to stone (10 ft movement 16 AC) \\
29-30 & Caster has light spell centered on self \\
31-32 & Caster can understand animals for the next hour \\
33-34 & Caster can only whisper for the next hour\\
35-36 & Goddess Nut takes interest in the caster and makes their next roll at advantage \\
37-38 & Random combatant shrinks to 1/2 size for the next hour\\
39-40 & Up to 8 nearby sticks turn to snakes\\
41-42 & 4d6 Fireball centered on caster\\
43-44 & Phantasmal Force of Rocks fall on all in range for 6d6\\
45-46 & 2 Random targets afflicted by confusion spell\\
47-48 & Target is deafened\\
49-50 & Caster temporarily grows working gills for next week\\
51-52 & 1000GP is teleported onto Caster..... take 1 damage\\
53-54 & Caster Instantly regains 10 HP\\
55-56 & Target Instantly regains 10HP\\
57-58 & Modron sent by the Arbiter appears\\
59-60 & Nearest monster above 8HD becomes acutely aware of caster\\
61-62 & Caster perceives surroundings as obscured by a dense fog (-2 to attacks)\\
63-64 & Caster perceives their own death\\
65-66 & All Allies of Caster with 30ft gain Flight for the next turn \\
67-68 & Nearby food and water spoils\\
69-70 & One random stat decreases by 1, one random stat increases by 1\\
71-72 & Caster instantly grows a massive beard\\
73-74 & Caster Temporarily forgets spell for the next week\\
75-76 & Sleep spell targeting randomly\\
77-78 & Allied Party falls under Bless spell\\
79-80 & Allied Party falls under Bane spell\\
81-82 & Caster's own shadow becomes a Shadow \\
83-84 & Caster forgets something important\\
85-86 & A loud bang occurs triggering a wandering monster \\
87-88 & Target becomes soaked in oil\\
89-90 & Target's legs turn into roots leaving them entangled for the next hour\\
91-92 & Target becomes immune to damage for the next round\\
93-94 & Target is teleported to another plane\\
95-96 & Ally falls under confusion spell\\
97-98 & Caster falls under reverse gravity\\
99 & Wild Magic Gauge immediately empty \\
00 & Choose \\
\end{tabularx}
\end{center}
\label{table:WildMagic}
\end{table}

\begin{multicols}{2}
\section*{Spells}%%%%%%%%%%%%%%%%%%%%%%%%%%%%%%%%Spells
\addcontentsline{toc}{section}{Spells}
\section*{Miracles}%%%%%%%%%%%%%%%%%%%%%%%%%%%%%%%%Miracles
\addcontentsline{toc}{section}{Miracles}
\end{multicols}
\section*{Equipment}%%%%%%%%%%%%%Equipment
\addcontentsline{toc}{section}{Equipment}
\label{section:Equipment}

\subsection*{Armor}%%%%%%%%%%%Armor
\addcontentsline{toc}{subsection}{Armor}
\begin{table}[H]

\begin{center}

\large
\rowcolors{2}{gray!30}{greg}

\begin{tabularx}{\textwidth}{b b b}

\hiderowcolors


 \textbf{
Armor}& \textbf{Armor Class} &\textbf{Cost}\\
\bottomrule
\bottomrule
\showrowcolors

Leather & 12 & 20\\
Chainmail & 14 & 40\\
Plate & 16 & 60\\
Shield & +1 & 10\\
\end{tabularx}
\end{center}
\label{table:Armor}
\end{table}%%%%%%%%%%%%%%End Armor

\subsection*{Weapons}%%%%%%%%%%%Weapon
\addcontentsline{toc}{subsection}{Weapons}
\begin{table}[H]
\begin{center}
\large
\rowcolors{2}{gray!30}{greg}
\begin{tabularx}{\textwidth}{X s s X X }
\hiderowcolors
 \centering{\textbf{
Weapon}}& \textbf{Damage}&\textbf{Cost (Gold)}& \multicolumn{1}{c}{\textbf{Properties}}\\
\hline
\hline
\showrowcolors
Battle axe & 1d8 & 8 &  Cutter\\
Club & 1d6 & 1 & Blunt\\
Crossbow & 1d6 & 25 & Reload, Range (50)\\
Dagger & 1d4 & 2 & Finesse, Range (20)\\
Greatsword & 1d10 & 15 & Two-Hand \\
Great Club & 1d8 & 6 & Blunt, Two-Hand \\
Hand axe & 1d6 & 3 & Cutter, Range (20)\\
Holy Water & 1d8 & 25 & Splash 5ft, Range (15), Holy\\
Javelin & 1d6 & 4 & Range(30)\\
Lance & 1d10 & 5 & Reach, Mounted\\
Longbow & 1d6 & 30 & Range(60) \\
Mace & 1d6 & 4 & \\
Oil Flask & 1d8 & 5 & Splash 5ft, Range(15), Fire\\
Pole Arm & 1d8 & 8 & Reach, Two-Hand\\
Short sword & 1d6 & 6 & \\
Sling & 1d4 & 2 & Blunt, Range (20)\\
Spear & 1d6 & 3 & Reach \\
Sword & 1d8 & 10 & \\
\end{tabularx}
\end{center}
\label{table:Weapons}
\end{table}%%%%%%%%%%%%%%End Weapon
\begin{multicols}{2}
\begin{mercHeading}
Properties
\end{mercHeading}
\begin{itemize}
\setlength\itemsep{0em}
	\item \textbf{Blunt} Usable by Clerics and other Holy men
	\item \textbf{Cutter} Multipurpose for cutting wood and breaking wooden doors (+1)
	\item \textbf{Finesse} Can replace attack roll attribute with \textbf{Dexterity}
	\item \textbf{Fire} Deals fire damage and can burn objects
	\item \textbf{Holy} Only affects Undead and Abyssal monsters
	\item \textbf{Mounted} More effective when mounted, -1 to attack roll when on foot
	\item \textbf{Range} Each increment of range (ft) decreases attack roll by 1. If not listed, assume 10
	\item \textbf{Reach} Allows the user to attack 10 ft
	\item \textbf{Two-Hand} Uses both hands
\end{itemize}
\begin{mercHeading}
Special Magic Weapon Rule
\end{mercHeading}
A weapon with \textbf{Two-Hand} property multiples magic damage bonus by 2. One hand free adds to attack roll.
\subsection*{Retainers}
\end{multicols}
\appendix
\section*{Terms}
\addcontentsline{toc}{section}{Term Reference}
\begin{tabular}{c c}

AC & Armor Class.... \\

TODO & TODO \\

\end{tabular}

\end{document}