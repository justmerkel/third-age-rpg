\documentclass[18pt]{article}

\usepackage{multicol}
\usepackage{color}
\usepackage[many]{tcolorbox}
\usepackage{lipsum}
\usepackage{setspace}
%\usepackage{float}
\usepackage{booktabs}
\usepackage{listings}
\usepackage{tabularx}
\usepackage{array}
\newcolumntype{P}[1]{>{\centering\arraybackslash}p{#1}}
\newcolumntype{Y}{>{\centering\arraybackslash}X}

\newcolumntype{b}{>{\centering\arraybackslash}X}
\newcolumntype{s}{>{\centering\arraybackslash\hsize=.5\hsize}X}
\newcolumntype{S}{>{\centering\arraybackslash\hsize=.4\hsize}X}
\usepackage{xcolor,colortbl}
\usepackage[a4paper, margin=0.5in,footskip=0.5in, bottom=1in]{geometry}
\tcbuselibrary{listings,breakable}
\usetikzlibrary{calc}

\definecolor{myblue}{RGB}{0,163,243}
\definecolor{greg}{RGB}{164,216,216}
\newcommand{\mhead}[1]{\paragraph{#1}\mbox{}\\\\}
\usepackage{titlesec}
\usepackage{lmodern}
\usepackage{booktabs}
\titleformat*{\section}{\Huge\bfseries}
\titleformat*{\subsection}{\LARGE\bfseries}
\titleformat*{\subsubsection}{\large\bfseries}
\titleformat*{\paragraph}{\large\bfseries}
\titleformat*{\subparagraph}{\large\bfseries}
\tcbset{mystyle/.style={
  breakable,
  enhanced,
  outer arc=0pt,
  arc=0pt,
  colframe=greg,
  colback=greg,
  attach boxed title to top left,
  boxed title style={
    colback=greg,
    outer arc=0pt,
    arc=0pt,
    },
  title=Example~\thetcbcounter,
  fonttitle=\sffamily
  }
}

\usepackage{floatrow}
\DeclareFloatFont{sans}{\sffamily \centering}% "scriptsize" is defined by floatrow, "tiny" not
\floatsetup[table]{font=sans}

\newtcolorbox{mercHeading}{
  width=\columnwidth,
  fontupper = \huge \sffamily,
  colback={greg},
boxsep=5pt,left=0pt,right=0pt,top=0pt,bottom=0pt,
  frame hidden,
  boxrule =0pt,
  arc = 0mm,
  outer arc =0mm,
  colupper = black
}

\newtcolorbox{mercClassInfo}{
  width=\columnwidth,
  fontupper = \large \sffamily,
  colback={greg},
boxsep=5pt,left=0pt,right=0pt,top=0pt,bottom=0pt,
  frame hidden,
  boxrule =0pt,
  arc = 0mm,
  outer arc =0mm,
  colupper = black
}



\title{\fontsize{70}{80} \selectfont The Third Age Rules Handbook}
\author{J. R. Merkel}
\date{}
\begin{document}
\sffamily
\large
\frenchspacing
\singlespacing
\parskip=0pt
\parindent=0pt
\lstset{aboveskip=0pt, belowskip=0pt}
%Paragraph header on newline
%\newcommand{\h2}[1]{\paragraph{#1}\mbox{}\\}


\maketitle
\thispagestyle{empty}
\pagebreak
\thispagestyle{empty}
\tableofcontents
\pagebreak
 \pagenumbering{arabic}
\section*{Introduction}
\addcontentsline{toc}{section}{Introduction}

\subsection*{Design Goals}
\begin{itemize}
	\item Maintain compatibility and tone from Old-school material.
        \item Revise pain points of B/X
        \item Incorporate 3 tier of outcomes from pbta
\end{itemize}
%%%%%%%%%%%Begin Actual Content%%%%%%%%%%%%%%%
\begin{multicols}{2}
\section*{Character}
\addcontentsline{toc}{section}{Character}
\subsection*{Creating a Character}
A Third Age Character has 5%TODO
steps.
\begin{enumerate}
	\item Generate attributes \emph{pg. \pageref{label:Attributes}}
	\item Decide on a class \emph{pg. \pageref{label:Class}}
	\item Roll hit points and starting gold, then fill out attack bonus, magic/divine, languages, and spells
	\item Purchase equipment and fill out Armor Class \emph{pg. \pageref{section:Equipment}}
	\item Create a name and backstory/personality %TODO Personality/Qualities from SR???
\end{enumerate}


\subsection*{Attributes}
\addcontentsline{toc}{subsection}{Attributes}
\label{label:Attributes}
The Third Age RPG uses 5 main stats that define a character's strengths and weaknesses ranging from -3 to +3. 

\begin{mercHeading}
Body
\end{mercHeading}
The measure of physical ability. The primary attribute of Fighting-men.

\begin{itemize}
	\setlength\itemsep{0em}
	\item \textbf{Hit Points:} Increases hit points by Body each level (Min 1)
	\item \textbf{Melee Attack Bonus:} Increases to-hit and damage rolls of melee attacks by Body
\end{itemize}

\begin{mercHeading}
Mind
\end{mercHeading}
The measure of mental acuity. The primary attribute of Magic-Users.

\begin{itemize}
\setlength\itemsep{0em}
	\item \textbf{Languages:} Increase the number of languages known by Mind
	\item \textbf{Spellcasting:} Increases spellcasting checks by Mind
\end{itemize}

\begin{mercHeading}
Spirit
\end{mercHeading}
The measure of will and belief. The primary attribute of Clerics

\begin{itemize}
\setlength\itemsep{0em}
	\item \textbf{Miracles:} Increases miracle checks by Spirit
\end{itemize}

\begin{mercHeading}
Charisma
\end{mercHeading}
The measure of personality and charm. The primary attribute of TBD? nothing

\begin{itemize}
\setlength\itemsep{0em}
	\item \textbf{Reaction:} Increase reaction roll result by Charisma
	\item \textbf{Retainers:} Increase max number of retainers and their morale check by Charisma
\end{itemize}

\begin{mercHeading}
Dexterity
\end{mercHeading}

\begin{itemize}
\setlength\itemsep{0em}
	\item \textbf{Armor Class:} Increase Armor Class (AC) by Dexterity
	\item \textbf{Ranged Attack Bonus:} Increases to-hit of ranged attacks by Dexterity
\end{itemize}
The measure of deftness and agility. The primary attribute of Thieves

\begin{mercHeading}
Assigning Attributes
\end{mercHeading}

Roll 3d6 for each of the 5 Attributes in order accoring to the following table. Note, this does produce more powerful characters than traditional OSR stat generation.


\begin{table}[H]

\begin{center}
\Large
\rowcolors{2}{gray!30}{greg}
\begin{tabular}{ c  c  }


\textbf{
Roll Result} &\textbf{ Bonus}\\
\bottomrule
\bottomrule

3 & -3 \\

4-5 & -2 \\

6-7 & -1 \\

8-9 & 0 \\

10-13 & +1 \\

14-16 & +2 \\

17-18 & +3 \\

\end{tabular}
\end{center}
\label{table:RollingAttributes}
\vspace{-1cm}
\end{table}

\begin{mercHeading} %TODO General rule section?? when to call for roll
Attribute Roll
\end{mercHeading}

An \textbf{Attribute Roll} is 2d6 + attribute. On a result of 6 or less, the roll is a failure. On a result of 7-9, the roll is a partial success. On a result of 10+ the roll is considered a success.\\
\textbf{Modifying Attribute Rolls} A referee may decide any attribute roll to be easier (add up to +3 to a result) or more difficult (subtract up to 3). \\
\textbf{Advantage/Disadvantage} Another way to modify rolls is to use advantage best 2 out of 3 d6 or disadvantage worst 2 out of 3 d6.


\section*{Class}
\addcontentsline{toc}{section}{Class}
\subsection*{Class List}
\label{label:Class}
\begin{itemize}
	\item Magic-User
	\item Fighting-Man
	\item Thief
	\item Cleric
\end{itemize}

\end{multicols}
\section*{Cleric}%%%%%%%%%%%%%%%CLERIC

\begin{multicols}{3}
\begin{mercClassInfo}
\textbf{Casting Attribute:} Spirit\\
\textbf{Armor:} All\\
\textbf{Weapons:} Weapons with Blunt Property
\end{mercClassInfo}
\end{multicols}

\begin{table}[H]

\begin{center}

\Large
\rowcolors{2}{greg}{gray!30}%%%TODO why is this backwards
\centering
\begin{tabularx}{\textwidth}{S S s S b b}

\hiderowcolors


 \textbf{
Level}& \textbf{XP} &\textbf{Hit Dice} &\textbf{Divine} &  \textbf{Max Miracle Level} & \textbf {Attack Bonus}\\
\bottomrule
\bottomrule
\showrowcolors
\centering
1 &0&  1d6  & 1 & 1 & 0\\

2 &2000&  2d6 & 2 & 1 & 1 \\

3 &4000& 3d6 & 3 & 2 &1\\

4 &8000& 4d6 & 4 & 2 & 2\\

5 &16000& 5d6 & 5 & 3 & 2\\

6 &32000& 6d6 &6 & 3 & 3\\

7 &64000& 7d6 & 7 & 4 & 3\\

\end{tabularx}
\end{center}
\label{table:Cleric}
\end{table}

\begin{multicols}{2}%%%Cleric data

\begin{mercHeading}
Divine Spellcasting
\end{mercHeading}

See  \textbf{pg \pageref{section:GenMagicRules} Magic} for general spellcasting rules. \\
\textbf{Miracles Known:} A first-level Cleric knows a number of miracles equal to \textbf{Spirit} (Min 1.) This number is split between 0th level miracles (cantrips) and 1st level miracles. When a magic user gains a level, they learn an additional miracle of a level less than or equal to the max miracle level of the table. \vspace{2pt}\\ 
\textbf{Miracle Scrolls:}
A Cleric is able to cast any Miracle scroll (For every level above max miracle level adds -1 to the attribute roll). Additionally, a Cleric can learn a miracle from a scroll after spending 1 day per spell level in prayer.

\begin{mercHeading}
Turn Undead
\end{mercHeading}
A Cleric has a special ability with their holy symbol that can repel or destroy Undead creatures.\\
TODO TABLE

\end{multicols}

\section*{Thief}%%%%%%%%%%%%%%%%%%%%%THIEF
\begin{multicols}{3}
\begin{mercClassInfo}
\textbf{Casting Attribute:} None\\
\textbf{Armor:} Light\\
\textbf{Weapons:} All
\end{mercClassInfo}
\end{multicols}
\begin{table}[H]

\begin{center}

\Large
\rowcolors{2}{gray!30}{greg}
\centering
\begin{tabularx}{\textwidth}{s s s b b}

\hiderowcolors


 \textbf{
Level}& \textbf{XP} &\textbf{Hit Dice} & \textbf {Attack Bonus} & \textbf{Thief Expertise}\\
\bottomrule
\bottomrule
\showrowcolors
\centering
1 &0&  1d6  & 1 & 1\\

2 &1500&  2d6 & 1 & 1 \\

3 &3000& 3d6 & 2 & 1\\

4 &6000& 4d6 & 2 & 2\\

5 &12000& 5d6 & 3 & 2\\

6 &24000& 6d6 & 3 & 2\\

7 &48000& 7d6 & 4 & 3\\

\end{tabularx}
\end{center}
\label{table:Thief}
\end{table}

\begin{multicols}{2} %%% Thief Data

\begin{mercHeading}
Backstab 
\end{mercHeading}
When the thief successfully sneaks behind an enemy unseen, they attack with +4 bonus to hit and deal double damage.

\begin{mercHeading}
Thief Expertise
\end{mercHeading}
A thief character adds \textbf{Thief Expertise} to any attribute roll that involves the following
\begin{itemize}
	\setlength\itemsep{0em}
	\item Lockpick
	\item Hiding
	\item Moving Silently
	\item Climbing
	\item Pickpocketing
	
\end{itemize}

\end{multicols}

\section*{Fighting-Man}%%%%%%%%%%%%%%%%%%%%Fighting-Man
\begin{multicols}{3}
\begin{mercClassInfo}
\textbf{Casting Attribute:} None\\
\textbf{Armor:} All\\
\textbf{Weapons:} All
\end{mercClassInfo}
\end{multicols}

\subsubsection*{Level Progression}
\vspace{-10pt}

\begin{table}[H]

\begin{center}

\Large
\rowcolors{2}{gray!30}{greg}
\centering
\begin{tabularx}{\textwidth}{s s s s}

\hiderowcolors


 \textbf{
Level}& \textbf{XP} &\textbf{Hit Dice} & \textbf {Attack Bonus}\\
\bottomrule
\bottomrule
\showrowcolors
\centering
1 &0&  1d8  & 1 \\

2 &2000&  2d8 & 2  \\

3 &4000& 3d8 & 3 \\

4 &8000& 4d8 & 4 \\

5 &16000& 5d8 & 5 \\

6 &32000& 6d8 &6 \\

7 &64000& 7d8 & 7 \\

\end{tabularx}
\end{center}
\label{table:Fighter}
\end{table}

\begin{multicols}{2}
You're a chad who doesn't need abilities %%%TODO WHAT IS A FIGHTER
\end{multicols}

\section*{Magic-User}%%%%%%%%%%%%%%%%%MAGIC USER
\addcontentsline{toc}{section}{Magic-User}
\begin{multicols}{3}
\begin{mercClassInfo}
\textbf{Casting Attribute:} Mind\\
\textbf{Armor:} None\\
\textbf{Weapons:} Dagger, Crossbow
\end{mercClassInfo}
\end{multicols}
\subsubsection*{Level Progression}
\vspace{-10pt}

\begin{table}[H]

\begin{center}

\Large
\rowcolors{2}{gray!30}{greg}
\centering
\begin{tabularx}{\textwidth}{s s s s b b}

\hiderowcolors


 \textbf{
Level}& \textbf{XP} &\textbf{Hit Dice} &\textbf{Magic} &  \textbf{Max Spell Level} & \textbf {Attack Bonus}\\
\bottomrule
\bottomrule
\showrowcolors
\centering
1 &0&  1d4  & 1 & 1 & 0\\

2 &2500&  2d4 & 2 & 1 & 0 \\

3 &5000& 3d4 & 3 & 2 &1\\

4 &10000& 4d4 & 4 & 2 & 1\\

5 &20000& 5d4 & 5 & 3 & 1\\

6 &40000& 6d4 &6 & 3 & 2\\

7 &80000& 7d4 & 7 & 4 & 2\\

\end{tabularx}
\end{center}
\label{table:MagicUser}
\end{table}
\begin{multicols}{2}
\begin{mercHeading}
Arcane Spellcasting
\end{mercHeading}
See  \textbf{pg \pageref{section:GenMagicRules} Magic} for general spellcasting rules. \\
\textbf{Spells Known:} A first-level magic user knows a number of spells equal to \textbf{Mind} (Min 1.) This number is split between 0th level spells (cantrips) and 1st level spells. When a magic user gains a level, they learn an additional spell of a level less than or equal to the max spell level of the table. \vspace{2pt}\\ 
\textbf{Spell Scrolls:}
A magic user is able to cast any spell scroll. Additionally, a magic user can learn a spell from a scroll after spending 1 day per spell level in study.\\

\section*{General}
TODO
\begin{mercHeading}
Multiclassing
\end{mercHeading}
\section*{Encounters}
TODO
\begin{mercHeading}
Reaction Rolls
\end{mercHeading}
\section*{Dungeon}
TODO
\begin{mercHeading}
Dungeon Turn
\end{mercHeading}
\begin{mercHeading}
Random Encounters
\end{mercHeading}
\section*{Combat}
TODO
\begin{mercHeading}
Initiative 
\end{mercHeading}
\begin{mercHeading}
Movement
\end{mercHeading}
\begin{mercHeading}
Attack Rolls
\end{mercHeading}
\section*{Magic}
\label{section:GenMagicRules}
Every character capable of casting spells or miracles has a \textbf{Magic} or \textbf{Divine} stat. This stat acts as maximum amount of wild magic energy a character can withstand before being unable to cast. That energy is expressed as a temperature gauge (known as \textbf{Wild Magic Gauge}) on a character's sheet with a number of segments equal to the stat. If a spell would exceed this gauge on a failure, the spell cannot be cast.
\begin{mercHeading}
{Adding to Wild Magic Gauge}
\end{mercHeading}
When a character casts a spell from a given class, they must make an attribute roll of the class's casting attribute. On a failure, the spell fizzles and has either no or reduced effect, and the spell's level is added to the wild magic gauge. On a partial success, the spell executes as normal, but the spell's level is added to the wild magic gauge. On a success, the spell executes as normal. Cantrips are a special type of spell that do not add to the gauge on a partial success.\vspace{2pt}\\
  \textbf{Full Wild Magic Gauge:}
Once the wild magic gauge is filled, the spell has an additional wild magic effect. This is determined by rolling on the Wild Magic table on pg \pageref{table:WildMagic} or an effect the referee thinks is thematic.
\begin{mercHeading}
Concentration
\end{mercHeading}
Spells that specify concentration or require multiple rounds to cast, uses the caster's action every round until the spell ends or is cast. When a caster is hit while concentrating, they must make a \textbf{Body} attribute roll to maintain the spell. On a failure, the spell fizzles and ends. On a partial success, the caster can choose to roll on the wild magic table or end the spell. On a success, nothing happens.

\section*{Equipment}
\label{section:Equipment}
\end{multicols}
\appendix
\section*{Terms}
\begin{tabular}{|c |c|}
\hline
AC & Armor Class.... \\
\hline
TODO & TODO \\
\hline
\end{tabular}
\end{document}